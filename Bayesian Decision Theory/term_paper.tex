\documentclass[letterpaper,12pt]{article}
%% Always use 12pt - it is much easier to read
%% Things written after '%' sign, are ignored by the latex editor - they are how to
%% Anything mathematics related should be put in between '$' signs.
%% Set some names and numbers here so we can use them below
\newcommand{\myname}{} %%%%%%%%%%%%%%% ---------> 
\newcommand{\mynumber}{Caden Hewlett}
\newcommand{\hw}{} 
%% There is a bit of stuff below which you should not have to change
%%%%%%
%% AMS mathematics packages - they contain many useful fonts and symbols.
\usepackage{amsmath, amsfonts, amssymb, array, xcolor}
%% The geometry package changes the margins to use more of the page, I suggest
%% using it because standard latex margins are chosen for articles and letters,
%% not homework.
\usepackage[paper=letterpaper,left=25mm,right=25mm,top=3cm,bottom=25mm]{geometry}
%% For details of how this package work, google the ``latex geometry
%%
%% Fancy headers and footers - make the document look nice
\usepackage{fancyhdr} %% for details on how this work, search-engi
\usepackage{framed}
\usepackage{mathalpha}
\usepackage{natbib}

\pagestyle{fancy}
%%
%% The header
\lhead{Philosophy 321} % course name as top-left
\chead{Term Paper} % homework number in top-centre
\rhead{ \myname \\ \mynumber }
%% This is a little more complicated because we have used `` \\ '' to force a line
%%
%% The footer
\lfoot{\myname} % name on bottom-left
\cfoot{Page \thepage} % page in middle
\rfoot{\mynumber} % student number on bottom-right
%%
%% These put horizontal lines between the main text and header and footer.
\renewcommand{\headrulewidth}{0.4pt}
\renewcommand{\footrulewidth}{0.4pt}
%%%
% Some useful macros
\usepackage{amsmath,amssymb,amsthm}
\usepackage{enumerate}
\usepackage{enumitem}
\usepackage{upgreek}
\usepackage{bbm}
\usepackage{hyperref}
\usepackage{setspace}
\newcommand{\ZZ}{\mathbb{Z}}
\usepackage{placeins}
\newcommand{\FF}{\mathbb{F}}
\newcommand{\RR}{\mathbb{R}}
\newcommand{\QQ}{\mathbb{Q}}
\newcommand{\CC}{\mathbb{C}}
\newcommand{\NN}{\mathbb{N}}
\newcommand{\ddx}{\dfrac{\text{d}}{\text{d}x}}
\renewcommand\vec{\mathbf}
\newcommand{\st}{\text{ s.t. } }
\newcommand{\dee}[1]{\mathrm{d}#1}
\newcommand{\diff}[2]{ \frac{\dee{#1}}{\dee{#2}} }
\newcommand{\lt}{<}
\newcommand{\gt}{>}
\newcommand{\set}[1]{\left\{#1 \right\}}
\newcommand{\dig}[1]{\left\langle{#1}\right\rangle}
\newcommand{\closure}[1]{\overline{#1}}
\newcommand{\interior}[1]{\mathrm{int}\left(#1\right)}
\newcommand{\boundary}[1]{\delta\left(#1\right)}
\newcommand{\ceiling}[1]{\left\lceil #1 \right\rceil}
\newcommand{\given}{|}

% Begin document
\begin{document}
	\doublespacing
	
\textit{Introduction}: In progress...

\textit{Background}: To introduce the nomenclature, we let $\mathcal{X}$ be the observation space, representing all possible data you might observe. Further, we let ${\Theta}$ be the parameter space, capturing all conceivable ``states of nature." Finally, we let $\mathcal{A}$ be the action space, consisting of all actions or estimators you can choose in response to the observed data. For this paper, an action $a \in \mathcal{A}$ generally implies estimating $\theta \in \Theta$ with a certain formula $\upalpha(x)$ known as the \textit{estimator}. For simplicity, we allow all of these spaces to be discrete. The observations $x \in \mathcal{X}$ are connected to the parameter $\theta \in \Theta$ by the probability mass function $p(x \given \theta)$, referred to as the data-generating process (DGP) \cite{tu2004data}. In a discrete setting, the DGP describes the probability of an observation $x \in \mathcal{X}$ under a given parameter $\theta$. The primary objective of statistical inference is to infer underlying properties of the DGP \cite{upton2008oxford}. From a decision-theoretic perspective, the decision $a \in \mathcal{A}$ will propose a function $\upalpha(x)$ to estimate the parameter $\theta$ as precisely as possible. To illustrate this concept, suppose we are flipping a fair coin and wish to recover the parameter $\theta$ corresponding to the proportion of heads, $\theta = 0.5$. Thus, the DGP $p(x \given \theta)$ is a Bernoulli distribution with parameter $\theta$. One action $a_1 \in \mathcal{A}$ is to propose the estimator $\upalpha_1(x) = \frac{1}{n}\sum_{i = 1}^n x_i$ (where $n$ is the number of flips) whereas $a_2 \in \mathcal{A}$ is to naively propose $\upalpha_2 (x) = 1$ (every flip is heads). It can be shown\footnote{ Given $p(x\given \theta) = \prod_{i = 1}^n\theta^{x_i}(1-\theta)^{1-x_i}$, the log-likelihood of the $n$ observations is $\ell(x, \theta) = \log(\theta) \sum_{i = 1}^n x_i + \log(1 - \theta) \sum_{i = 1}^n (1 - x_i)$. Maximizing wrt $\theta$ yields $\hat{\theta}_{\text{MLE}} = \frac{1}{n}\sum_{i = 1}^n x_i = \upalpha_1(x)$.} that $a_1$ proposes an estimator which maximizes the likelihood of the observed data under the DGP \cite{rossi2018}, whereas $a_2$'s estimator is biased, thus trivially $a_1 \succ a_2$. Unless necessarily distinct, we henceforth use estimators $\upalpha$ and the actions $a$ proposing them interchangeably. 

To quantify the preference orderings beyond the simple heuristics mentioned in the coin-flipping case, statisticians leverage loss functions  \cite{wald1950}, which we denote $\mathcal{L}(\theta, \upalpha)$. The loss function represents the error associated with proposing a ``bad" estimation of the $\theta$ (or function of $\theta$) of interest. Thus, the best evaluation of this function is a zero loss; therefore, $\mathcal{L}(\theta, \upalpha) \geq 0$ \cite{robert2007bayesian}. From a decision-theoretic perspective, the objective of the decision-maker is to propose an estimator $\upalpha$ which minimizes this loss. Since the actual value of parameter $\theta$ is often unknown, statisticians base their ordering of estimators on the \textit{expected} loss. However, precisely how we define \textit{expected} relies upon whether one takes a {frequentist} or {Bayesian} approach. 

\textit{Frequentism and Minimax}: Under the frequentist paradigm, the data $x \in \mathcal{X}$ are considered random because they arise from repeated sampling via the DGP $p(x \given \theta)$. Meanwhile, $\theta$ is treated as a fixed but unknown constant in the parameter space $\Theta$. In the coin-flipping example, a frequentist would assume that the coin has a fixed (unknown) probability $\theta$ of landing heads, and thus $p(x\given \theta)$ governs each flip outcome. Thus, to evaluate a proposed estimator $\upalpha$, the frequentist approach focuses on expected loss, akin to how Peterson \cite{peterson2017} considers the expected utility. Specifically, we define the expected loss (EL) as the product of the probability of observing $x \in \mathcal{X}$ and the loss associated with estimating $\theta$ with $\upalpha(x)$,
\begin{equation}
	\text{EL}(\theta, \upalpha) = \mathbb{E}_{\theta}[\mathcal{L}(\theta, \upalpha)] = \sum_{x \in \mathcal{X}} \mathcal{L}\big( \theta, \upalpha(x) \big) p(x \given \theta)  \label{eq:EL}
\end{equation}
Other works refer to the above as a risk function \cite{nikulin2001}. From this definition of expected loss, we introduce the concept of ``minimax" through a game-theoretic analogy of a game against Nature. In this framework, our goal is to select an estimator $\upalpha \in \mathcal{A}$ that \textit{minimizes} our expected loss. Meanwhile, Nature acts as an adversary, selecting a parameter $\theta \in \Theta$ (i.e., a ``state of the world") in an attempt to \textit{maximize} our expected loss \cite{ulansky2021}. The expected loss in such a game is known as the ``minimax risk", which we define as
\begin{equation}
	\overline{{R}} = \min_{\upalpha \in \mathcal{A}} \max_{\theta \in \Theta} \text{EL}(\theta, \upalpha) \label{eq:minimaxrisk}
\end{equation}
The minimax estimator is known as the estimator/decision rule $\upalpha \in \mathcal{A}$ that achieves the minimax risk. While the minimax risk $\overline{R}$ is occasionally criticized as being overly conservative \cite{peterson2017}, the ability of an estimator to be the best in the worst case scenario (which we refer to as the ``minimax guarantee") is desirable for many real-world applications including management of financial portfolios \cite{DENG2005278}.

Having defined the minimax risk in \hyperref[eq:minimaxrisk]{Equation (2)} and the corresponding guarantee, we now turn to \textit{The Bayesian Choice} \cite{robert2007bayesian}, in which Christian Robert demonstrates that under the ``least favourable" prior, Bayesian decision theory achieves a Bayes risk that is at least as good (and often better than) the frequentist minimax bound. We first introduce the notion of a \textit{prior} to explain the Bayesian paradigm. In a discrete parameter space, the prior is a function $\uppi: \Theta \mapsto [0,1] \subseteq \mathbb{R}$ satisfying $\sum_{\theta \in \Theta} \uppi (\theta) = 1$ where $\uppi(\theta)$ is the probability that $\theta$ is the ``true" state of the world. Vitally, the data $x \in \mathcal{X}$ still arise from the DGP $p(x \given \theta)$ (now known as the ``likelihood") but are treated as \textit{fixed} once observed. Bayesian methods instead place uncertainty in $\theta$, initially via $\uppi(\theta)$ and later in $\uppi(\theta \given x)$, referred to as the ``posterior," once observing the data $x$. In contrast, frequentist methods conceptualize $x \in \mathcal{X}$ as potentially variable under repeated sampling, while $\theta$ is fixed but unknown.


From the perspective of Decision Theory, to evaluate a proposed estimator $\upalpha \in \mathcal{A}$, the Bayesian approach focuses on the posterior expected loss (PEL), which averages the loss associated with estimating $\theta$ with $\upalpha(x)$ across all possible values of $\theta \in \Theta$, where the average is weighted by the posterior probability of the parameter $\uppi(\theta \given x)$ conditioned on the observed value $x$. 
\begin{equation}
	\text{PEL}(\theta, \upalpha \given x) = \mathbb{E}_{\uppi}[\mathcal{L}(\theta, \upalpha)\given x] = \sum_{\theta \in \Theta} \mathcal{L}(\theta, \upalpha(x)) \uppi(\theta \given x) \label{eq:EPL}
\end{equation}
The equation above considers the weighted loss across all $\theta \in \Theta$ for a singular $x$, whereas \hyperref[eq:EL]{Equation (1)} weighs across all $x \in \mathcal{X}$ for a singular $\theta$; thus, the two measures are not necessarily commensurable. To allow for direct comparison with the Bayesian framework to the frequentist paradigm, Roberts introduces\footnote{Note that formal definitions of Bayes Risk date as far back as the 1980s with seminal works from James O. Berger\cite{berger1985}. Roberts himself cites these works as part of his argument.} the notion of \textit{Bayes Risk}

Importantly, the Bayesian framework is not necessarily incommensurable with the frequentist paradigm: the combined notion of \textit{Bayes risk} is crucial in Robert's preference of Bayesian decision theory over its frequentist counterpart. 

% "not necessarily as dangerous as frequentists may depict it"
\textit{Remaining Work}
\begin{enumerate}\item{Introduce Robert's Argument and proof. (Bayesianism, Integrated Risk, proof of Integrated Risk $\leq$ Minimax Risk using weighted sum vs. set maxima)} 
\item{Introduce Stark's Counterargument: How the prior $\pi(\theta)$ is subjective, and Robert's proof is trivial since you are ``adding information" to the risk problem which was previously constrained by objectivity.}
\item{Introduce the Bayesian Rebuttal: Namely, the subjectivity of choice of loss function $\mathcal{L}(\dots)$ implies the frequentist construction of the problem isn't operating under such ``objective constraints," so given that subjective claims need to be made on the state of Nature, a Bayesian approach gives provable optimality. }
\item{Conclusion and Introduction }
 \end{enumerate}
 %Formally, by examining how integrated Bayesian risk can match the worst-case criterion, Robert’s analysis shows that Bayesian decision theory is “at least as good” as the frequentist minimax approach, yet retains the interpretive benefits of a prior-based framework.
%% transition into Christian Robert’s book The Bayesian Choice, in which he uses minimaxity of integrated bayesian risk as an argument in favour of a Bayesian decision-theoretic approach over a frequentist one. 
%  Christian Robert’s book The Bayesian Choice
% Roberts' section:
% Intro to bayesian philosophy (what is random what isn't)
% Connection to coin example again
% Introduce integrated risk (maybe posterior expected loss)
%% Frequentist risk averaged over the values of theta according to their posterior distribution
% Introduce Bayes Estimator (Definition 2.3.3)
%% and Bayes risk as r(\pi) = r(\pi, \alpha_{\text{B}})
%% r(\pi, \alpha) = \sum_{\theta \in \Theta}\sum_{x \in \mathcal{X}} \mathcal{L}\big(\theta, \alpha(x) \big)p_{\mathcal{X}\mid\Theta}(x \mid \theta) p_{\Theta}(\theta)
%% or using pi notation
%% \alpha_{\text{B}} = \underset{ {\alpha \in \mathcal{A}} }{\text{argmin}}\sum_{\theta \in \Theta}\sum_{x \in \mathcal{X}} \mathcal{L}\big(\theta, \alpha(x) \big)p_{\mathcal{X}\mid\Theta}(x \mid \theta) \pi(\theta) \\
% Demonstrate using least favourable prior that
% the Bayes risk is always smaller than the minmax risk
%% \max_{\pi \in \Pi}r(\pi, \alpha_{\text{B}}) \leq R(\theta, \overline{\alpha}_{\text{F}})
% using weighted sum <= maximum
% making a decision based on the expected value of the loss function; 
%  represents the loss (or error) owing to a bad evaluation of the function of θ of interest, and therefore that even the best evaluation of this function, i.e.,
% when θ is known, can induce at best a null loss.
\bibliographystyle{unsrt}
\bibliography{final_paper_bib}  
% Introduce Minimax Risk
%% Introduce minimax estimator
%% \overline{\alpha}_{\text{F}} = \text{argmin}_{\alpha \in \mathcal{A}} \max_{\theta \in \Theta}\sum_{x \in \mathcal{X}}\mathcal{L}(\theta , \alpha(x))f(x\mid \theta)
%% "minimax guarantee" in the worst-case scenario


\end{document}

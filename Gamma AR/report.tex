% Options for packages loaded elsewhere
% Options for packages loaded elsewhere
\PassOptionsToPackage{unicode}{hyperref}
\PassOptionsToPackage{hyphens}{url}
\PassOptionsToPackage{dvipsnames,svgnames,x11names}{xcolor}
%
\documentclass[
  letterpaper,
  DIV=11,
  numbers=noendperiod]{scrartcl}
\usepackage{xcolor}
\usepackage{amsmath,amssymb}
\setcounter{secnumdepth}{-\maxdimen} % remove section numbering
\usepackage{iftex}
\ifPDFTeX
  \usepackage[T1]{fontenc}
  \usepackage[utf8]{inputenc}
  \usepackage{textcomp} % provide euro and other symbols
\else % if luatex or xetex
  \usepackage{unicode-math} % this also loads fontspec
  \defaultfontfeatures{Scale=MatchLowercase}
  \defaultfontfeatures[\rmfamily]{Ligatures=TeX,Scale=1}
\fi
\usepackage{lmodern}
\ifPDFTeX\else
  % xetex/luatex font selection
\fi
% Use upquote if available, for straight quotes in verbatim environments
\IfFileExists{upquote.sty}{\usepackage{upquote}}{}
\IfFileExists{microtype.sty}{% use microtype if available
  \usepackage[]{microtype}
  \UseMicrotypeSet[protrusion]{basicmath} % disable protrusion for tt fonts
}{}
\makeatletter
\@ifundefined{KOMAClassName}{% if non-KOMA class
  \IfFileExists{parskip.sty}{%
    \usepackage{parskip}
  }{% else
    \setlength{\parindent}{0pt}
    \setlength{\parskip}{6pt plus 2pt minus 1pt}}
}{% if KOMA class
  \KOMAoptions{parskip=half}}
\makeatother
% Make \paragraph and \subparagraph free-standing
\makeatletter
\ifx\paragraph\undefined\else
  \let\oldparagraph\paragraph
  \renewcommand{\paragraph}{
    \@ifstar
      \xxxParagraphStar
      \xxxParagraphNoStar
  }
  \newcommand{\xxxParagraphStar}[1]{\oldparagraph*{#1}\mbox{}}
  \newcommand{\xxxParagraphNoStar}[1]{\oldparagraph{#1}\mbox{}}
\fi
\ifx\subparagraph\undefined\else
  \let\oldsubparagraph\subparagraph
  \renewcommand{\subparagraph}{
    \@ifstar
      \xxxSubParagraphStar
      \xxxSubParagraphNoStar
  }
  \newcommand{\xxxSubParagraphStar}[1]{\oldsubparagraph*{#1}\mbox{}}
  \newcommand{\xxxSubParagraphNoStar}[1]{\oldsubparagraph{#1}\mbox{}}
\fi
\makeatother


\usepackage{longtable,booktabs,array}
\usepackage{calc} % for calculating minipage widths
% Correct order of tables after \paragraph or \subparagraph
\usepackage{etoolbox}
\makeatletter
\patchcmd\longtable{\par}{\if@noskipsec\mbox{}\fi\par}{}{}
\makeatother
% Allow footnotes in longtable head/foot
\IfFileExists{footnotehyper.sty}{\usepackage{footnotehyper}}{\usepackage{footnote}}
\makesavenoteenv{longtable}
\usepackage{graphicx}
\makeatletter
\newsavebox\pandoc@box
\newcommand*\pandocbounded[1]{% scales image to fit in text height/width
  \sbox\pandoc@box{#1}%
  \Gscale@div\@tempa{\textheight}{\dimexpr\ht\pandoc@box+\dp\pandoc@box\relax}%
  \Gscale@div\@tempb{\linewidth}{\wd\pandoc@box}%
  \ifdim\@tempb\p@<\@tempa\p@\let\@tempa\@tempb\fi% select the smaller of both
  \ifdim\@tempa\p@<\p@\scalebox{\@tempa}{\usebox\pandoc@box}%
  \else\usebox{\pandoc@box}%
  \fi%
}
% Set default figure placement to htbp
\def\fps@figure{htbp}
\makeatother





\setlength{\emergencystretch}{3em} % prevent overfull lines

\providecommand{\tightlist}{%
  \setlength{\itemsep}{0pt}\setlength{\parskip}{0pt}}



 


\KOMAoption{captions}{tableheading}
\makeatletter
\@ifpackageloaded{caption}{}{\usepackage{caption}}
\AtBeginDocument{%
\ifdefined\contentsname
  \renewcommand*\contentsname{Table of contents}
\else
  \newcommand\contentsname{Table of contents}
\fi
\ifdefined\listfigurename
  \renewcommand*\listfigurename{List of Figures}
\else
  \newcommand\listfigurename{List of Figures}
\fi
\ifdefined\listtablename
  \renewcommand*\listtablename{List of Tables}
\else
  \newcommand\listtablename{List of Tables}
\fi
\ifdefined\figurename
  \renewcommand*\figurename{Figure}
\else
  \newcommand\figurename{Figure}
\fi
\ifdefined\tablename
  \renewcommand*\tablename{Table}
\else
  \newcommand\tablename{Table}
\fi
}
\@ifpackageloaded{float}{}{\usepackage{float}}
\floatstyle{ruled}
\@ifundefined{c@chapter}{\newfloat{codelisting}{h}{lop}}{\newfloat{codelisting}{h}{lop}[chapter]}
\floatname{codelisting}{Listing}
\newcommand*\listoflistings{\listof{codelisting}{List of Listings}}
\makeatother
\makeatletter
\makeatother
\makeatletter
\@ifpackageloaded{caption}{}{\usepackage{caption}}
\@ifpackageloaded{subcaption}{}{\usepackage{subcaption}}
\makeatother
\usepackage{bookmark}
\IfFileExists{xurl.sty}{\usepackage{xurl}}{} % add URL line breaks if available
\urlstyle{same}
\hypersetup{
  pdftitle={Non-Normal Time Series Estimation},
  colorlinks=true,
  linkcolor={blue},
  filecolor={Maroon},
  citecolor={Blue},
  urlcolor={Blue},
  pdfcreator={LaTeX via pandoc}}


\title{Non-Normal Time Series Estimation}
\author{}
\date{}
\begin{document}
\maketitle


\subsection{Introduction}\label{introduction}

Consider a first-order autoregressive time series
\(X_t = \varphi X_{t-1} + \varepsilon_t\), where the marginal
distribution is non-normal and \(\{\varepsilon_t\}\) is a sequence of
\emph{iid} random variables with characteristic function given by the
the Characteristic Function / Laplace-Stieltjes transform:

\[
\Phi_{\varepsilon}(s) = \dfrac{\Phi_X(s)}{\Phi_X(\varphi s)}
\]

Where \(\Phi_X(s)\) is the Characteristic Function (CF) of the
stationary non-normal series \(\{X\}_t\).

\subsubsection{Gamma AR(1) Process}\label{gamma-ar1-process}

In this work, there is an estimation method for the gamma AR(1) process
of Sim (1990). It replaces \(\varphi X_{t-1}\) of the original model
with \(\varphi \ast X_{t-1}\), where \('\ast'\) is the discrete sum:

\[
\varphi \ast X = \sum_{i = 1}^{N(X)} W_i
\] Where \(W_i \overset{\text{iid}}{\sim} \text{Exp}(\beta)\) (rate
parameter \(\beta\)) and for each fixed value of \(x\),
\(N(x) \sim \text{Pois}(\alpha \beta x)\).

Then, the AR(1) gamma model places marginal distribution
\(\text{Gamma}(\nu, \beta(1-\varphi))\) on each \(X\), given by

\[
X_t = \varphi \ast X + \varepsilon_t
\]

In the above, \(\{\varepsilon_t\}\) are gamma-distributed with rate
\(\beta\) and shape \(\nu\).

The marginal density of \(\{X_t\}\) is given by:

\[
f_{X}(x_t) = \dfrac{1}{\Gamma(\nu)} \big(\beta(1 - \varphi) \big)^{\nu}x_t^{\nu - 1}\exp\Big(-\beta(1-\varphi)x_t \Big)
\] This is found directly from the gamma PDF.

Further, the conditional density of \(X_{t+j}\) given \(X_{t}\) is given
by\ldots{}

\[
f_{X_{t+j} \mid X_t}(x \mid y) = \theta \Big( \dfrac{x}{\varphi^jy}\Big)^{(\nu -1)/2} \exp\Big(-\theta(x + \varphi^j y) \Big)I_{\nu - 1}(2 \theta (\varphi^j x y)^{1/2})
\] Where \(\theta = \beta(1-\varphi)/(1-\varphi^j)\),
\(\varphi \in (0,1)\) and \(I_{r}(z)\) is the modified Bessel function
of the first kind and of order \(r\), given by

\[
I_r(z) = \sum_{k = 0}^{\infty} \dfrac{1}{k!\Gamma(k + r + 1)}\Big(\frac{z}{2} \Big)^{2k + r}
\] Which in this case is of the form\ldots{} \[
I_{\nu - 1}(2 \theta (\varphi^j x y)^{1/2}) = \sum_{k = 0}^{\infty} \dfrac{1}{k!\Gamma(k + \nu)}\big(\theta (\varphi^j x y)^{1/2}\big)^{2k + \nu -1}
\]

\subsubsection{Model Estimation}\label{model-estimation}

\[
\mathscr{L}(\boldsymbol{\theta}; x_0, x_1, \dots x_n) = \log(f_X(x_0)) + \sum_{i = 1}^n \log f_{X \mid X_{t-1}}(x_t \mid x_{t-1})
\]

\paragraph{Calculations}\label{calculations}

\[
\begin{aligned}
\Phi_X(s) &= \mathbb{E}[\exp(-sX_n)] = \mathbb{E}[\exp(-s(\varphi X_{n-1} + \varepsilon_n))]  \\
&= \mathbb{E}[\exp(-s\varphi X_{n-1})\exp(-s\varepsilon_n)] \\
&= \mathbb{E}[\exp(-s\varphi X_{n-1})]\mathbb{E}[\exp(-s\varepsilon_n)] \\
&= \Phi_X(\varphi s)\Phi_{\varepsilon}(s)
\end{aligned}
\]

Now, consider

\$\$

\begin{aligned}
X_n &= \varphi \ast X_{n-1} + \varepsilon_n\\
\text{ Where: } \\
\varepsilon_n &\overset{\text{iid}}{\sim} \text{Gamma}(\varphi, \nu) \\

\varphi \ast X &= \sum_{i = 1}^{N(X)}W_i \\
W_i &\overset{\text{iid}}{\sim} \text{Exponential}(\varphi)\\
N(X) \mid X =x &\sim \text{Poisson}(\lambda = p\varphi), \;\; p \in [0,1) \\
\end{aligned}

\$\$

In this case, what is crucial is that for a fixed \(X\), the poisson pmf
is still multiplied by \(x\) since it is a poisson Process, i.e.

\[
\mathbb{P}[N(X) = n \mid X =X] = \frac{(\lambda x)^n}{n!}\exp(- \lambda x)
\] From the above, the Laplace transform of \(\varphi \ast X\) for a
fixed \(X = x\) can be computed by recalling that the sum of independent
random variables is equal to the convolution of their probability
distributions.
(\href{https://en.wikipedia.org/wiki/Laplace\%E2\%80\%93Stieltjes_transform}{source}
)

\[
\begin{aligned}
\{\mathcal{L}^{*}_{\varphi \ast X}(s) &= \sum_{n = 0}^{\infty} \mathbb{P}[N(x) = n]\big(\mathbb{E}[-sW]\big)^n \\
&= \sum_{n = 0}^{\infty}  \frac{(\lambda x)^n}{n!}\exp(- \lambda x)\big(M_W(-s))^n \\
&= \sum_{n = 0}^{\infty}  \frac{(\lambda x)^n}{n!}\exp(- \lambda x) \Big( \dfrac{\varphi}{\varphi + s} \Big)^n \\
&= \exp(- \lambda x)\sum_{n = 0}^{\infty}  \frac{1}{n!} \Big( \dfrac{\lambda x\varphi}{\varphi + s} \Big)^n \\
\end{aligned}
\] Finally, using the power series expansion of the exponential
function, \[
\exp(a) = \sum_{n = 0}^{\infty} \frac{1}{n!}a^n
\]

The author's result follows: \[
\begin{aligned}
\{\mathcal{L}^{*}_{\varphi \ast X}(s) \}&= \exp(-\lambda x)\exp\Big( \dfrac{\lambda x\varphi}{\varphi + s} \Big) \\
&= \exp\big(-\lambda x  (1-\dfrac{\varphi}{\varphi + s} ) \big) \\
&= \boxed{\exp\Big( \dfrac{-\lambda x s}{\varphi + s} \Big)}
\end{aligned}
\]

From this construction, the authors establish equation 2.2, the Laplace
transform of a singular \(X_n\) conditioned on a fixed prior observation
\(X_{n-1}\). The below follows from the independence of
\(\varepsilon_t\) from the cumulative process \(\varphi \ast X_{n-1}\),
yields

\[
\begin{aligned}
\mathbb{E}[\exp(-sX_n) \mid X_{n-1} = x] &= \mathbb{E}[\exp\big(-s( \varphi \ast X_{n-1} + \varepsilon_n) \big) \mid X_{n-1} = x] \\
&= \mathbb{E}[-s\varepsilon_t]\mathbb{E}[-s(\varphi \ast X_{n-1})\mid X_{n-1} = x] \\
&= \boxed{ \Big(\dfrac{\varphi}{s + \varphi}\Big)^\nu\exp\Big( \dfrac{-\lambda x s}{\varphi + s} \Big)} \;\; \text{ result 2.2}
\end{aligned}
\]

Then, the authors allow \(\Phi_{X_n}(s)\) to be the Laplace transform of
the PDF of \(X_n\), i.e. \[
\Phi_{X_n}(s) = \mathbb{E}[\exp(-sX_n)], \text{ for } s \geq 0
\] From the above (conditional) LT, it follows by the Law of Total
Expectation that \[
\begin{aligned}
\mathbb{E}[\exp(-sX_n)] &= \mathbb{E}\Big[ \mathbb{E}[\exp(-sX_n) \mid X_{n-1}] \Big] \\
\Phi_{X_n}(s) &= \mathbb{E}\Big[ \Big(\dfrac{\varphi}{s + \varphi}\Big)^\nu\exp\Big( \dfrac{-\lambda X_{n-1} s}{\varphi + s} \Big) \Big] \\
&=  \Big(\dfrac{\varphi}{s + \varphi}\Big)^\nu\mathbb{E}\Big[\exp\Big( -X_{n-1} \dfrac{\lambda  s}{\varphi + s} \Big) \Big] \\
&=  \Big(\dfrac{\varphi}{s + \varphi}\Big)^\nu\Phi_{X_{n-1}} \Big( \dfrac{\lambda  s}{\varphi + s} \Big) \\
&=  \Big(\dfrac{\varphi}{s + \varphi}\Big)^\nu \Phi_{X_{n-1}} \Big( \dfrac{\varphi p  s}{\varphi + s} \Big), \text{ recalling } \lambda = \varphi p
\end{aligned}
\]




\end{document}
